\documentclass[]{elsarticle} %review=doublespace preprint=single 5p=2 column
%%% Begin My package additions %%%%%%%%%%%%%%%%%%%
\usepackage[hyphens]{url}



\usepackage{lineno} % add
\providecommand{\tightlist}{%
  \setlength{\itemsep}{0pt}\setlength{\parskip}{0pt}}

\usepackage{graphicx}
%%%%%%%%%%%%%%%% end my additions to header

\usepackage[T1]{fontenc}
\usepackage{lmodern}
\usepackage{amssymb,amsmath}
\usepackage{ifxetex,ifluatex}
\usepackage{fixltx2e} % provides \textsubscript
% use upquote if available, for straight quotes in verbatim environments
\IfFileExists{upquote.sty}{\usepackage{upquote}}{}
\ifnum 0\ifxetex 1\fi\ifluatex 1\fi=0 % if pdftex
  \usepackage[utf8]{inputenc}
\else % if luatex or xelatex
  \usepackage{fontspec}
  \ifxetex
    \usepackage{xltxtra,xunicode}
  \fi
  \defaultfontfeatures{Mapping=tex-text,Scale=MatchLowercase}
  \newcommand{\euro}{€}
\fi
% use microtype if available
\IfFileExists{microtype.sty}{\usepackage{microtype}}{}
\bibliographystyle{elsarticle-harv}
\usepackage{longtable,booktabs,array}
\usepackage{calc} % for calculating minipage widths
% Correct order of tables after \paragraph or \subparagraph
\usepackage{etoolbox}
\makeatletter
\patchcmd\longtable{\par}{\if@noskipsec\mbox{}\fi\par}{}{}
\makeatother
% Allow footnotes in longtable head/foot
\IfFileExists{footnotehyper.sty}{\usepackage{footnotehyper}}{\usepackage{footnote}}
\makesavenoteenv{longtable}
\ifxetex
  \usepackage[setpagesize=false, % page size defined by xetex
              unicode=false, % unicode breaks when used with xetex
              xetex]{hyperref}
\else
  \usepackage[unicode=true]{hyperref}
\fi
\hypersetup{breaklinks=true,
            bookmarks=true,
            pdfauthor={},
            pdftitle={Something related to Bayesian data analysis},
            colorlinks=false,
            urlcolor=blue,
            linkcolor=magenta,
            pdfborder={0 0 0}}
\urlstyle{same}  % don't use monospace font for urls

\setcounter{secnumdepth}{5}
% Pandoc toggle for numbering sections (defaults to be off)

% Pandoc citation processing

% Pandoc header
\usepackage{booktabs}
\usepackage{longtable}
\usepackage{array}
\usepackage{multirow}
\usepackage{wrapfig}
\usepackage{float}
\usepackage{colortbl}
\usepackage{pdflscape}
\usepackage{tabu}
\usepackage{threeparttable}
\usepackage{threeparttablex}
\usepackage[normalem]{ulem}
\usepackage{makecell}
\usepackage{xcolor}



\begin{document}
\begin{frontmatter}

  \title{Something related to Bayesian data analysis}
      
  \begin{abstract}
  
  \end{abstract}
  
 \end{frontmatter}

\section{Methods}\label{methods}

The cut-off of biomarker AI and its predictive values were determined
using a Bayesian model described by Vradi et al. (2018). The model can
be expressed as:\\
\[Y|X \sim Bernoulli(p)\] \[
p(x) = P(Y = 1|X = x) = 
\left\{
    \begin{array}{lr}
          p_1 = P(Y = 1|X \le cp) = 1 - NPV, & \text{if } x \le cp\\
          p_2 = P(Y = 1|X > cp) = PPV, & \text{if } x > cp
    \end{array}
\right.
\]\\
The binary response variable \emph{Y} took the value 1 when a patient
had NPSLE and 0 otherwise. \emph{X} was the continuous measurement of
the biomarker AI. In this study, the positive predictive value (PPV) of
the cutoff \emph{cp} was expected to \(\ge\) 70\%, while the negative
predictive value (NPV) was \(\ge\) 50\%. Thus, prior distributions of
1-NPV and PPV were chosen as follows: \[p_1 \sim Uniform(0, 0.5)\]
\[p_2 \sim Uniform(0.7, 1)\] The Bayesian analysis was carried out in
SAS Studio (SAS Institute Inc., 2015) through PROC MCMC, which uses
Markov chain Monte Carlo (MCMC) algorithm. The model was run for a total
of \ensuremath{2\times 10^{6}} iterations. The first \ensuremath{10^{6}}
iterations were discarded as burn-in. The remaining were kept 1 in 10
samples (thinning) to reduce autocorrelation. The MCMC
representativeness and accuracy were assessed by trace plot, sample
autocorrelation, effective sample size (ESS), and Monte Carlo standard
error (MCSE).

\section{Results}\label{results}

\begin{itemize}
\tightlist
\item
  Diagnostic results

  \begin{itemize}
  \tightlist
  \item
    Trace plot of all parameters mixed well =\textgreater{} showed no
    potential problem with convergence.\\
  \item
    Sample autocorrelation of all parameters dropped quickly to zero
    with increasing lag.\\
  \item
    ESS of all parameters were \textgreater{} 17500. (recommended:
    \textgreater{} 10000)\\
  \item
    MCSE of all parameters were \textless{} 0.0059.\\
    =\textgreater{} Conclusion: MCMC chains were stable and accurate.
  \end{itemize}
\item
  Posterior distribution summary\\
\item
  report: the mean and 95\% highest density interval (HDI) of cp, p1,
  p2. (in Table \textbf{Posterior Summaries and Intervals} of the file
  \textbf{Results\_cutoff\_VEleni\_final.pdf})
\end{itemize}

\begin{longtable}[]{@{}lll@{}}
\caption{\label{tab:unnamed-chunk-2}this table is used to compare mean and
median value when necessary}\tabularnewline
\toprule
Parameter & Median & Mean\tabularnewline
\midrule
\endfirsthead
\toprule
Parameter & Median & Mean\tabularnewline
\midrule
\endhead
cp & 5.1522 & 5.259981\tabularnewline
p1 & 0.4471 & 0.439842\tabularnewline
p2 & 0.8800 & 0.870422\tabularnewline
\bottomrule
\end{longtable}

\begin{itemize}
\tightlist
\item
  95\% HDI of cutoff value was in range 3.6796-7.1657 but 89\% HDI of
  cutoff was in this small range 4.2134-5.9498.
\end{itemize}

\section{Caption of Figure:}\label{caption-of-figure}

Fig1. Summary of posterior distribution of the cutoff and its predictive
value (1-NPV and PPV). The vertical red dotted lines denote the median
of the distribution.

\section{Supplementary}\label{supplementary}

\subsection{Diagnostic results}\label{diagnostic-results}

\subsection{SAS MCMC code}\label{sas-mcmc-code}

\begin{verbatim}
PROC import datafile="/path/to/data.csv"
DBMS=csv out=Data replace;
RUN; 


PROC MCMC 
    data=Data outpost=Dataoutput 
        nbi=1000000 
        nmc=1000000
        thin=10
        seed=1
        diag=all
        monitor=(p1 p2 cp I w); 
PARMS cp1 cp2 p1 p2 w I; 
prior cp1 ~ uniform(1,10); 
prior cp2 ~ normal(5,sd=1); 
hyperprior I ~ beta(1,1); 
prior w ~ binary(I); 
cp = w*cp1 + (1-w)*cp2; 
prior p2 ~ uniform(0.7, 1);
prior p1 ~ uniform(0, 0.5);
p = (AI<=cp)*p1 + (AI>cp)*p2; 
model Y ~ binary(p);
RUN;
\end{verbatim}

\section*{References}\label{references}
\addcontentsline{toc}{section}{References}

\hypertarget{refs}{}
\hypertarget{ref-SAS2015}{}
SAS Institute Inc., 2015. SAS OnDemand for Academics: User's Guide 524.

\hypertarget{ref-Vradi2018}{}
Vradi, E., Jaki, T., Vonk, R., Brannath, W., 2018. A Bayesian model to
estimate the cutoff and the clinical utility of a biomarker assay.
Statistical Methods in Medical Research 1--19.
doi:\href{https://doi.org/10.1177/0962280218784778}{10.1177/0962280218784778}


\end{document}

